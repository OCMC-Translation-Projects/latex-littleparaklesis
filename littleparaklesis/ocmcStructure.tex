% Author: Michael Colburn
% specifies packages and defines new commands for purposes
% of typesetting liturgical books for the Eastern Orthodox Christian Church
% 
% the prefix 'lt' means 'liturgical text'.  This prefix was added to commands
% to avoid collisions with commands named by other packages.
%
% This system uses key-value pairs for the language files.  LaTeX does
% not allow non-alpha characters as the key name.  You can avoid this
% problem by putting them in quotes, e.g. "p001".  
% The value (text is not put in  quotes)
%

% Author: Michael Colburn
% Purpose: controls various options for typesetting liturgical text

%==================================================
% Show keys?
%
% Controls whether the PDF will display the key used to retrieve the
% text.
%
%\def\showKeys{} % comment out to NOT display keys
% 
%==================================================

%==================================================
% Show line numbers?
%
% Controls whether the PDF will a number for each line.
%
%\def\showLineNumbers{} % comment out to NOT display numbers
% 
%==================================================

%==================================================
% Actors on separate line?
%
% Controls whether the actor will appear on its on line.
% For example, 
%
% If enabled:
% Priest: In the name of the Father and of the Son...
%
% If not enabled:
% Priest
% In the name of the Father and of the Son...
%
%\def\actorOnSeparateLine{} % comment out to NOT display keys
% 
%==================================================

%==================================================
% One language or two?
% 
% Controls whether we will have one language (monolingual)
% or two languages (bilingual)
% Bilingual text is typeset as two columns on each page.
%
%\newcommand*{\createbilingual}{} % comment out to do monolingual
%
%==================================================

%==================================================
% the name of the directory contain the language tex files
\def\ltResourceDir{resources/}
% 
%==================================================

%==================================================
% languages
\def\ltLeftMain{"en_UK_lash"}
\def\ltLeftAlt{"en_KE_oak"}
\def\ltLeftLextionaryMain{"en_US_rsv"}
\def\ltLeftLextionaryAlt{"en_US_saas"}
% 
%==================================================

%==================================================
% Which language or languages?
% 
% Controls language or languages to use.
% Each language must have a folder with that name.
% If you are typesetting just one language (monolingual)
% then comment out the \def\langRight
%
\def\langLeft{\ltLeftMain} % if monolingual, only this language will be used
\def\langRight{"gr_gr_cog"} 
%
%==================================================
 % control.tex allows you to easily change settings

%================================================
% Packages
%================================================
\usepackage[protrusion=false]{microtype}
\usepackage{color}
\usepackage{xcolor}
\definecolor{ocre}{RGB}{243,102,25} % Define the orange color used for highlighting throughout the book
\definecolor{auburn}{rgb}{0.43, 0.21, 0.1}


\usepackage{titlesec}
\usepackage{etoolbox}
\usepackage{microtype} % Slightly tweak font spacing for aesthetics
\usepackage[T1]{fontenc} % Output font encoding for international characters
\usepackage[osf]{libertine} % Use the Libertine font
\usepackage{xifthen}
\usepackage{catchfilebetweentags} % for doing key based look ups of text
\usepackage{parskip}
\usepackage{titling}
\usepackage{trimspaces}

%----------------------------------------------------------------------------------------
% Need to add the following back into master copy of OCMC Structure
%----------------------------------------------------------------------------------------

\usepackage{graphicx}
\graphicspath{ {images/} }
\usepackage{lineno}

% The following 4 lines are for Chinese, Japanese, and Korean
\usepackage{xeCJK}
\setCJKmainfont{UnGungseo.ttf}
\setCJKsansfont{UnGungseo.ttf}
\setCJKmonofont{gulim.ttf}

%----------------------------------------------------------------------------------------
% End of Section to be added back into master copy of OCMC Structure
%----------------------------------------------------------------------------------------


%----------------------------------------------------------------------------------------
%	HYPERLINKS IN THE DOCUMENTS
%----------------------------------------------------------------------------------------

\usepackage{hyperref}
\hypersetup{
  hidelinks
  ,backref=true
  ,pagebackref=true
  ,hyperindex=true
  ,colorlinks=true
  ,citecolor=blue
  ,filecolor=blue
  ,linkcolor=blue
  ,urlcolor= ocre
  ,breaklinks=true
  ,bookmarks=true
  ,bookmarksopen=false
  ,pdftitle={Title}
  ,pdfauthor={Author}
}
\usepackage{bookmark}
\bookmarksetup{
open,
numbered,
addtohook={%
\ifnum\bookmarkget{level}=0 % chapter
\bookmarksetup{bold}%
\fi
\ifnum\bookmarkget{level}=-1 % part
\bookmarksetup{color=ocre,bold}%
\fi
}
}

%=========================================================
% Global Settings
%=========================================================
\setlength{\parindent}{0em} % turns off indenting of paragraphs.  Comment out to turn on.
\setlength{\parskip}{.5em}  % spacing between paragraphs
\setsecnumdepth{none}
\maxsecnumdepth{none}
\titleformat{\section}[block]{\color{blue}\Large\bfseries\filcenter}{}{1em}{}
\pagestyle{ruled}
\makeevenhead{ruled}{}{\rightmark}{\thepage}
\makeoddhead{ruled}{\thepage}{\leftmark}{}
\makeevenfoot{ruled}{}{}{}
\makeoddfoot{ruled}{}{}{}

%=========================================================
% Custom Commands
%=========================================================
\newcommand\asteriskfill{\leavevmode\xleaders\hbox{$\ast$}\hfill\kern0pt}

\newcommand\ltSwitchResource[1]{\renewcommand\langLeft{#1}}
\newcommand\ltSwitchToDefault{\renewcommand\langLeft{\ltLeftMain}}

\newcommand\ltSid[2]{\ExecuteMetaData[\ltResourceDir\langLeft/"#1"]{#2} \ifdef{\showKeys}{\tiny #1:#2\normalsize}{}} % to look up values by an id

\newcommand\ltActor[1]{\ifdef{\actorOnSeparateLine}{\noindent\textcolor{red}{#1}\newline}{\noindent\textcolor{red}{\ignorespaces#1\unskip:}}}
\newcommand\ltBlack[1]{#1}
\newcommand\ltCover[1]{\huge{#1}}
\newcommand\ltDesignation[1]{#1}
\newcommand\ltDialog[1]{#1}
\newcommand\ltHymn[1]{#1}
\newcommand\ltHymnlinelast[1]{#1}
\newcommand\ltInaudible[1]{#1}
\newcommand\ltTitle[1]{#1}
\newcommand\ltMelody[1]{#1}
\newcommand\ltMixed[1]{#1}
\newcommand\ltMode[1]{#1}
\newcommand\ltReading[1]{#1}
\newcommand\ltRed[1]{\textcolor{red}{#1}}
\newcommand\ltRubric[1]{\textcolor{red}{#1}}
\newcommand\ltVerse[1]{#1}

